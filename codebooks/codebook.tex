% Options for packages loaded elsewhere
\PassOptionsToPackage{unicode}{hyperref}
\PassOptionsToPackage{hyphens}{url}
%
\documentclass[
]{scrartcl}
\usepackage{amsmath,amssymb}
\usepackage{lmodern}
\usepackage{iftex}
\ifPDFTeX
  \usepackage[T1]{fontenc}
  \usepackage[utf8]{inputenc}
  \usepackage{textcomp} % provide euro and other symbols
\else % if luatex or xetex
  \usepackage{unicode-math}
  \defaultfontfeatures{Scale=MatchLowercase}
  \defaultfontfeatures[\rmfamily]{Ligatures=TeX,Scale=1}
\fi
% Use upquote if available, for straight quotes in verbatim environments
\IfFileExists{upquote.sty}{\usepackage{upquote}}{}
\IfFileExists{microtype.sty}{% use microtype if available
  \usepackage[]{microtype}
  \UseMicrotypeSet[protrusion]{basicmath} % disable protrusion for tt fonts
}{}
\makeatletter
\@ifundefined{KOMAClassName}{% if non-KOMA class
  \IfFileExists{parskip.sty}{%
    \usepackage{parskip}
  }{% else
    \setlength{\parindent}{0pt}
    \setlength{\parskip}{6pt plus 2pt minus 1pt}}
}{% if KOMA class
  \KOMAoptions{parskip=half}}
\makeatother
\usepackage{xcolor}
\IfFileExists{xurl.sty}{\usepackage{xurl}}{} % add URL line breaks if available
\IfFileExists{bookmark.sty}{\usepackage{bookmark}}{\usepackage{hyperref}}
\hypersetup{
  pdftitle={Codebook bundeslaendeR},
  pdfauthor={Robert Stelzle},
  hidelinks,
  pdfcreator={LaTeX via pandoc}}
\urlstyle{same} % disable monospaced font for URLs
\usepackage[lmargin=2.5cm, rmargin=3.5cm, tmargin=2.5cm, bmargin=2.5cm,
includeheadfoot]{geometry}
\usepackage{color}
\usepackage{fancyvrb}
\newcommand{\VerbBar}{|}
\newcommand{\VERB}{\Verb[commandchars=\\\{\}]}
\DefineVerbatimEnvironment{Highlighting}{Verbatim}{commandchars=\\\{\}}
% Add ',fontsize=\small' for more characters per line
\usepackage{framed}
\definecolor{shadecolor}{RGB}{248,248,248}
\newenvironment{Shaded}{\begin{snugshade}}{\end{snugshade}}
\newcommand{\AlertTok}[1]{\textcolor[rgb]{0.94,0.16,0.16}{#1}}
\newcommand{\AnnotationTok}[1]{\textcolor[rgb]{0.56,0.35,0.01}{\textbf{\textit{#1}}}}
\newcommand{\AttributeTok}[1]{\textcolor[rgb]{0.77,0.63,0.00}{#1}}
\newcommand{\BaseNTok}[1]{\textcolor[rgb]{0.00,0.00,0.81}{#1}}
\newcommand{\BuiltInTok}[1]{#1}
\newcommand{\CharTok}[1]{\textcolor[rgb]{0.31,0.60,0.02}{#1}}
\newcommand{\CommentTok}[1]{\textcolor[rgb]{0.56,0.35,0.01}{\textit{#1}}}
\newcommand{\CommentVarTok}[1]{\textcolor[rgb]{0.56,0.35,0.01}{\textbf{\textit{#1}}}}
\newcommand{\ConstantTok}[1]{\textcolor[rgb]{0.00,0.00,0.00}{#1}}
\newcommand{\ControlFlowTok}[1]{\textcolor[rgb]{0.13,0.29,0.53}{\textbf{#1}}}
\newcommand{\DataTypeTok}[1]{\textcolor[rgb]{0.13,0.29,0.53}{#1}}
\newcommand{\DecValTok}[1]{\textcolor[rgb]{0.00,0.00,0.81}{#1}}
\newcommand{\DocumentationTok}[1]{\textcolor[rgb]{0.56,0.35,0.01}{\textbf{\textit{#1}}}}
\newcommand{\ErrorTok}[1]{\textcolor[rgb]{0.64,0.00,0.00}{\textbf{#1}}}
\newcommand{\ExtensionTok}[1]{#1}
\newcommand{\FloatTok}[1]{\textcolor[rgb]{0.00,0.00,0.81}{#1}}
\newcommand{\FunctionTok}[1]{\textcolor[rgb]{0.00,0.00,0.00}{#1}}
\newcommand{\ImportTok}[1]{#1}
\newcommand{\InformationTok}[1]{\textcolor[rgb]{0.56,0.35,0.01}{\textbf{\textit{#1}}}}
\newcommand{\KeywordTok}[1]{\textcolor[rgb]{0.13,0.29,0.53}{\textbf{#1}}}
\newcommand{\NormalTok}[1]{#1}
\newcommand{\OperatorTok}[1]{\textcolor[rgb]{0.81,0.36,0.00}{\textbf{#1}}}
\newcommand{\OtherTok}[1]{\textcolor[rgb]{0.56,0.35,0.01}{#1}}
\newcommand{\PreprocessorTok}[1]{\textcolor[rgb]{0.56,0.35,0.01}{\textit{#1}}}
\newcommand{\RegionMarkerTok}[1]{#1}
\newcommand{\SpecialCharTok}[1]{\textcolor[rgb]{0.00,0.00,0.00}{#1}}
\newcommand{\SpecialStringTok}[1]{\textcolor[rgb]{0.31,0.60,0.02}{#1}}
\newcommand{\StringTok}[1]{\textcolor[rgb]{0.31,0.60,0.02}{#1}}
\newcommand{\VariableTok}[1]{\textcolor[rgb]{0.00,0.00,0.00}{#1}}
\newcommand{\VerbatimStringTok}[1]{\textcolor[rgb]{0.31,0.60,0.02}{#1}}
\newcommand{\WarningTok}[1]{\textcolor[rgb]{0.56,0.35,0.01}{\textbf{\textit{#1}}}}
\usepackage{graphicx}
\makeatletter
\def\maxwidth{\ifdim\Gin@nat@width>\linewidth\linewidth\else\Gin@nat@width\fi}
\def\maxheight{\ifdim\Gin@nat@height>\textheight\textheight\else\Gin@nat@height\fi}
\makeatother
% Scale images if necessary, so that they will not overflow the page
% margins by default, and it is still possible to overwrite the defaults
% using explicit options in \includegraphics[width, height, ...]{}
\setkeys{Gin}{width=\maxwidth,height=\maxheight,keepaspectratio}
% Set default figure placement to htbp
\makeatletter
\def\fps@figure{htbp}
\makeatother
\setlength{\emergencystretch}{3em} % prevent overfull lines
\providecommand{\tightlist}{%
  \setlength{\itemsep}{0pt}\setlength{\parskip}{0pt}}
\setcounter{secnumdepth}{-\maxdimen} % remove section numbering
\usepackage{longtable}
\usepackage{float}
\usepackage{graphicx}
\usepackage{pdflscape}
\newcommand{\blandscape}{\begin{landscape}}
\newcommand{\elandscape}{\end{landscape}}
\usepackage{fontspec}
\usepackage{libertinus}
\setmonofont[Scale=MatchLowercase,FakeStretch=0.9]{Consolas}
\setkomafont{captionlabel}{\bfseries\small\sffamily}
\setkomafont{caption}{\small\sffamily}
\usepackage[automark, headsepline=1pt, footsepline=1pt]{scrlayer-scrpage}
\ohead{Codebook bundeslaendeR}
\ofoot*{23.09.2021} \ihead{\headmark}
\setkomafont{pagenumber}{\itshape}
\ifoot*{Seite \pagemark}
\chead{}
\cfoot*{}
\usepackage{booktabs}
\usepackage{longtable}
\usepackage{array}
\usepackage{multirow}
\usepackage{wrapfig}
\usepackage{float}
\usepackage{colortbl}
\usepackage{pdflscape}
\usepackage{tabu}
\usepackage{threeparttable}
\usepackage{threeparttablex}
\usepackage[normalem]{ulem}
\usepackage{makecell}
\usepackage{xcolor}
\ifLuaTeX
  \usepackage{selnolig}  % disable illegal ligatures
\fi
\usepackage[backref,style=ext-authoryear-comp,minnames=1,maxnames=3,isbn=false,date=year,sorting=nyt]{biblatex}
\addbibresource{literature.bib}

\title{Codebook bundeslaendeR}
\author{Robert Stelzle}
\date{21.01.2022}

\begin{document}
\maketitle

{
\setcounter{tocdepth}{1}
\tableofcontents
}
\noindent

\rule{\linewidth}{1pt}
\vspace{1.5cm}

\hypertarget{introduction}{%
\section{Introduction}\label{introduction}}

Hallo dies ist ein Test. \newpage

\begin{table}
\centering
    \normalsize
        \caption{State-level Variables}
        \label{statevars}
        
\tiny
\texttt{
\begin{tabular}{llll}
\toprule
\textbf{state} & \textbf{nuts1} & \textbf{state\_name\_de} & \textbf{state\_name\_en} \\
\midrule
BA & NA & ehemaliges Land Baden & former state Baden \\
BB & DE4 & Brandenburg & Brandenburg \\
BE & DE3 & Berlin & Berlin \\
BW & DE1 & Baden-Württemberg & Baden-Württemberg \\
BY & DE2 & Bayern & Bavaria \\
HB & DE5 & Bremen & Bremen \\
HE & DE7 & Hessen & Hesse \\
HH & DE6 & Hamburg & Hamburg\\
MV & DE8 & Mecklenburg-Vorpommern & Mecklenburg-Vorpommern\\
NI & DE9 & Niedersachsen & Lower-Saxony \\
NW & DEA & Nordrhein-Westfalen & North Rhine-Westphalia\\
RP & DEB & Rheinland-Pfalz & Rhineland-Palatine\\
SH & DEF & Schleswig-Holstein & Schleswig-Holstein\\
SL & DEC & Saarland & Saarland \\
SN & DED & Sachsen & Saxony\\
ST & DEE & Sachsen-Anhalt & Saxony-Anhalt\\
TH & DEG & Thüringen & Thuringia\\
WB & NA & ehemaliges Land Württemberg-Baden & former state Württemberg-Baden\\
WH & NA &ehemaliges Land Württemberg-Hohenzollern & former state Württemberg-Hohenzollern\\
\bottomrule
\end{tabular}}
\end{table}

\hypertarget{ltw_election_results}{%
\section{\texorpdfstring{\texttt{ltw\_election\_results}}{ltw\_election\_results}}\label{ltw_election_results}}

\texttt{bundeslaendeR::ltw\_election\_results} returns data frame
(tibble if the tibble package is loaded) containing one row per
contesting party per election. For a schematic version of
\texttt{bundeslaendeR::ltw\_election\_results}'s structure see
table\(~\)\ref{strltwelecres}.

Most election results data are provided by the Bundeswahlleiter. A
machine-readable version of the Bundeswahlleiter's compiled data
contained in the -periodically published- pdf available here
(\url{https://www.bundeswahlleiter.de/service/landtagswahlen.html}) was
kindly provided to me. Election data outside the timeframe covered by
Bundeswahlleiter's data provided to me was collected from the states'
local election authorities' (Landeswahlleiter) websites. More
information on parties and the continuity of parties under different
labels was collected by me.

The Bundeswahlleiter's election data in many cases contains differing
names for the same party. Both between states (eg. ``Christlich
Demokratische Union Deutschlands'' vs.~``Christlich Demokratische Union
Deutschlands in Niedersachsen'') as well as within states between
elections -in many cases due to parties being renamed- (``BÜNDNIS 90/DIE
GRÜNEN, Landesverband Hamburg, Grün-Alternative Liste'' vs.~``BÜNDNIS
90/DIE GRÜNEN, Landesverband Hamburg''). Efforts were made to reconcile
both of these inconsistencies by adding two new, harmonized variables
identifying parties (\texttt{partyname\_short} and \texttt{partyname}).
This harmonized party identifier also covers merging of parties. The
partyname given to the resulting party (eg. ``Linke'', ``Grüne'') is
given to the largest of the preceding parties contesting an election
unless a smaller party joined a government following the election. The
original names provided by the Bundeswahlleiter (and Landeswahlleiters
in elections after June 2021) are still available
(\texttt{partyname\_short\_bundeswahlleiter} and
\texttt{partyname\_bundeswahlleiter}).

\begin{table}
\centering
\caption{Structure of \texttt{ltw\_election\_results}}
\label{strltwelecres}
\tiny
\texttt{
\begin{tabular}{llrllrllrllr}
            \toprule
            \multicolumn{3}{c}{\textbf{\textsf{State Variables}}} & \multicolumn{3}{c}{\textbf{\textsf{Election Variables}}} & \multicolumn{3}{c}{\textbf{\textsf{Party Variables}}} & \multicolumn{3}{c}{\textbf{\textsf{Party-Election Variables}}}\\
            \multicolumn{3}{c}{\textsf{Name, Abbreviation, NUTS1 Code}} & \multicolumn{3}{c}{\textsf{Election date, Size Electorate, Turnout, ...}} & \multicolumn{3}{c}{\textsf{Names, Abbreviations, several IDs}} & \multicolumn{3}{c}{\textsf{Vote Count, -Share, Seat Count, -Share, ...}}\\
            \cmidrule(r){1-3}\cmidrule(lr){4-6}\cmidrule(lr){7-9}\cmidrule(l){10-12}
            \textbf{state} & \textbf{nuts1} & \textbf{...} & \textbf{election\_date} & \textbf{turnout} & \textbf{...} & \textbf{partyname\_short} & \textbf{ches\_id} & \textbf{...} & \textbf{party\_vshare} & \textbf{party\_seat\_count} & \textbf{...} \\
            \midrule
            BE & DE3 & ... & 2015-09-18 & 0.765 & ... & Party A & 001 & ... & 0.45 & 46 & ...\\
            BE & DE3 & ... & 2015-09-18 & 0.765 & ... & Party B & 002 & ... & 0.30 & 12 & ...\\
            BE & DE3 & ... & 2015-09-18 & 0.765 & ... & Party C & 003 & ... & 0.25 & 18 & ...\\
        &&&&&&&&&&&\\[-2ex]
            NI & DE9 & ... & 2012-12-16 & 0.560 & ... & Party A & 001 & ... & 0.17 & 12 & ...\\
            NI & DE9 & ... & 2012-12-16 & 0.560 & ... & Party B & 002 & ... & 0.33 & 27 & ...\\
            NI & DE9 & ... & 2012-12-16 & 0.560 & ... & Party D & 004 & ... & 0.50 & 46 & ...\\
            \bottomrule
    \end{tabular}}


\end{table}

\clearpage

\hypertarget{election-results-variables}{%
\subsection{Election Results
Variables}\label{election-results-variables}}

\begin{longtable}{p{3.2cm}| p{11cm}}
\texttt{state} &\textbf{State Abbreviation}\newline 
ISO 3166-2:DE-code of the state;
           including BA for the former state of Baden, WH for the former state
           of Württemberg-Hohenzollern and WB for the former state of
           Württemberg-Baden.
\end{longtable}

\begin{longtable}{p{3.2cm}| p{11cm}}
\texttt{nuts1} &\textbf{NUTS1 Code of State}\newline 
NUTS1 code of state. NA for former states Baden, Württemberg-Baden, Württemberg-Hohenzollern.
\end{longtable}

\begin{longtable}{p{3.2cm}| p{11cm}}
\texttt{state\_name\_de} &\textbf{German Name of State}\newline 
German name of the state.
\end{longtable}

\begin{longtable}{p{3.2cm}| p{11cm}}
\texttt{state\_name\_en} &\textbf{English Name of State.}\newline 
English name of the state.
\end{longtable}

\begin{longtable}{p{3.2cm}| p{11cm}}
\texttt{state\_election\_
term} &\textbf{Election Term of State}\newline 
Election term in the state. Counts up from 1.



\hspace*{.25cm}
\begin{minipage}[t]{\linewidth }
\vspace{0pt}
\includegraphics[width = \linewidth]{cbfiles/electiontermplot.pdf}
\end{minipage}



\end{longtable}

\begin{longtable}{p{3.2cm}| p{11cm}}
\texttt{election\_date} &\textbf{Election Date}\newline 
Date of the election.  ISO 8601 or R-Date format.






\hspace*{.25cm}
\begin{minipage}[t]{\linewidth }
\vspace{0pt}
\includegraphics[width = \linewidth]{cbfiles/electiondatesplot.pdf}
\end{minipage}






\end{longtable}

\begin{longtable}{p{3.2cm}| p{11cm}}
\texttt{election\_id\_
bundeswahlleiter} &\textbf{Election ID Bundeswahlleiter}\newline 
Specific election\_id as denoted by the Bundeswahlleiter. Note that BA, WH and WH are named as BW and the number counts down. NA for cases taken from Landeswahlleiters (i.e. elections after ST 2021).
\end{longtable}

\begin{longtable}{p{3.2cm}| p{11cm}}
\texttt{election\_remarks\_
bundeswahlleiter} &\textbf{Election Remarks Bundeswahlleiter}\newline 
Remarks on the election as given by the Bundeswahlleiter.
\end{longtable}

\begin{longtable}{p{3.2cm}| p{11cm}}
\texttt{electorate} &\textbf{Size of the Electorate}\newline 
Number of eligible voters. For more totals also see the last three columns.

\hspace*{.25cm}
\begin{minipage}[t]{\linewidth }
\vspace{0pt}
\includegraphics[width = \linewidth]{cbfiles/electorateplot.pdf}
\end{minipage}


\end{longtable}

\begin{longtable}{p{3.2cm}| p{11cm}}
\texttt{number\_of\_voters} &\textbf{Number of Voters}\newline 
Number of voters turning out. For more totals also see the last three columns.

\hspace*{.25cm}
\begin{minipage}[t]{\linewidth }
\vspace{0pt}
\includegraphics[width = \linewidth]{cbfiles/nvotersplot.pdf}
\end{minipage}


\end{longtable}

\begin{longtable}{p{3.2cm}| p{11cm}}
\texttt{turnout} &\textbf{Turnout}\newline 
Turnout. Share of eligible voters turning out.

\hspace*{.25cm}
\begin{minipage}[t]{\linewidth }
\vspace{0pt}
\includegraphics[width = \linewidth]{cbfiles/turnoutplot.pdf}
\end{minipage}


\end{longtable}

\begin{longtable}{p{3.2cm}| p{11cm}}
\texttt{valid\_votes} &\textbf{Valid Votes}\newline 
Number of valid votes. Does not have to be equal to the number of ballots cast, as sometimes a ballot contains multiple votes! For more totals also see the last three columns.

\hspace*{.25cm}
\begin{minipage}[t]{\linewidth }
\vspace{0pt}
\includegraphics[width = \linewidth]{cbfiles/validvoteplot.pdf}
\end{minipage}


\end{longtable}

\begin{longtable}{p{3.2cm}| p{11cm}}
\texttt{total\_seats\_
parliament} &\textbf{Total Seats in Parliament}\newline 
Total number of members of the newly elected Landtag.

\hspace*{.25cm}
\begin{minipage}[t]{\linewidth }
\vspace{0pt}
\includegraphics[width = \linewidth]{cbfiles/tseatsparlplot.pdf}
\end{minipage}


\end{longtable}

\begin{longtable}{p{3.2cm}| p{11cm}}
\texttt{female\_party\_
seats\_available} &\textbf{Number of female MdLs available per party}\newline 
Denotes whether information on the no. of female members of the Landtag per party is available for this election. Note that for parties not elected to the new Landtag party\_female\_mps always is.na() == TRUE.

\hspace*{.25cm}
\begin{minipage}[t]{\linewidth }
\vspace{0pt}
\includegraphics[width = \linewidth]{cbfiles/fpsaplot.pdf}
\end{minipage}


\end{longtable}

\begin{longtable}{p{3.2cm}| p{11cm}}
\texttt{total\_female\_
mps\_parliament} &\textbf{Number of Female MPs in Parliament}\newline 
Number of newly elected female MPs.

\hspace*{.25cm}
\begin{minipage}[t]{\linewidth }
\vspace{0pt}
\includegraphics[width = \linewidth]{cbfiles/totfemmpsplot.pdf}
\end{minipage}


\end{longtable}

\begin{longtable}{p{3.2cm}| p{11cm}}
\texttt{partyname\_short} &\textbf{Abbreviated Party Name}\newline 
Harmonized abbreviation of the party's name. 374 unique parties.
\end{longtable}

\begin{longtable}{p{3.2cm}| p{11cm}}
\texttt{partyname} &\textbf{Party Name}\newline 
Harmonized name of the party. 374 unique parties.
\end{longtable}

\begin{longtable}{p{3.2cm}| p{11cm}}
\texttt{partyname\_short\_
bundeswahlleiter} &\textbf{Party Name Abbreviation from Bundeswahlleiter}\newline 
Partyname abbreviation as documented by the Bundeswahlleiter. 461 different abbreviations.
\end{longtable}

\begin{longtable}{p{3.2cm}| p{11cm}}
\texttt{partyname\_
bundeswahlleiter} &\textbf{Party Name from Bundeswahlleiter}\newline 
Partyname as documented by the Bundeswahlleiter. 499 different names.
\end{longtable}

\begin{longtable}{p{3.2cm}| p{11cm}}
\texttt{party\_vote\_count} &\textbf{Party Vote Count}\newline 
Number of votes recieved by the party.

\hspace*{.25cm}
\begin{minipage}[t]{\linewidth }
\vspace{0pt}
\includegraphics[width = \linewidth]{cbfiles/pvcplot.pdf}
\end{minipage}


\end{longtable}

\begin{longtable}{p{3.2cm}| p{11cm}}
\texttt{party\_vshare} &\textbf{Party Vote Share}\newline 
Share of votes recieved by the party.

\hspace*{.25cm}
\begin{minipage}[t]{\linewidth }
\vspace{0pt}
\includegraphics[width = \linewidth]{cbfiles/pvsplot.pdf}
\end{minipage}


\end{longtable}

\begin{longtable}{p{3.2cm}| p{11cm}}
\texttt{party\_seat\_count} &\textbf{Party Seat Count}\newline 
Number of seats recieved by the party.

\hspace*{.25cm}
\begin{minipage}[t]{\linewidth }
\vspace{0pt}
\includegraphics[width = \linewidth]{cbfiles/pscplot.pdf}
\end{minipage}


\end{longtable}

\begin{longtable}{p{3.2cm}| p{11cm}}
\texttt{party\_sshare} &\textbf{Party Seat Share}\newline 
Share of seats recieved by the party.

\hspace*{.25cm}
\begin{minipage}[t]{\linewidth }
\vspace{0pt}
\includegraphics[width = \linewidth]{cbfiles/pssplot.pdf}
\end{minipage}


\end{longtable}

\begin{longtable}{p{3.2cm}| p{11cm}}
\texttt{party\_female\_mps} &\textbf{Number of female MPs from party}\newline 
Number of female MPs elected for the party. Note that for parties not elected to the new Landtag party\_female\_mps always is.na() == TRUE.

\hspace*{.25cm}
\begin{minipage}[t]{\linewidth }
\vspace{0pt}
\includegraphics[width = \linewidth]{cbfiles/pfmpcplot.pdf}
\end{minipage}


\end{longtable}

\begin{longtable}{p{3.2cm}| p{11cm}}
\texttt{wzb\_govelec\_id} &\textbf{WZB DD GovElec ID}\newline 
If available, MR-Code of the party in the internal govelec database of the WZB department Democracy and Democratization \parencite{wzbDatabasePartiesElections2021}. These party IDs are chiefly based on party IDs from \textcite{mackieInternationalAlmanacElectoral1991}.


\hspace*{.25cm}
\begin{minipage}[t]{\linewidth }
\vspace{0pt}
\includegraphics[width = \linewidth]{cbfiles/govelecplot.pdf}
\end{minipage}


\end{longtable}

\begin{longtable}{p{3.2cm}| p{11cm}}
\texttt{ches\_id} &\textbf{CHES ID}\newline 
If available, ID of the party in the Chapel-Hill Expert Survey \parencite{jollyChapelHillExpert2022}.


\hspace*{.25cm}
\begin{minipage}[t]{\linewidth }
\vspace{0pt}
\includegraphics[width = \linewidth]{cbfiles/chesplot.pdf}
\end{minipage}



\end{longtable}

\begin{longtable}{p{3.2cm}| p{11cm}}
\texttt{partyfacts\_id} &\textbf{PartyFacts ID}\newline 
If available, ID of the party in the partyfacts database \parencite{doringPartyFactsDatabase2019}.

\hspace*{.25cm}
\begin{minipage}[t]{\linewidth }
\vspace{0pt}
\includegraphics[width = \linewidth]{cbfiles/partyfactsplot.pdf}
\end{minipage}



\end{longtable}

\begin{longtable}{p{3.2cm}| p{11cm}}
\texttt{decker\_neu} &\textbf{Chapter Parteienhandbuch}\newline 
Denotes, wether the Handbuch der deutschen Parteien (3. ed.) by Decker and Neu \parencite{deckerHandbuchDeutschenParteien2018b} has a chapter on the party.

\hspace*{.25cm}
\begin{minipage}[t]{\linewidth }
\vspace{0pt}
\includegraphics[width = \linewidth]{cbfiles/deckerneuplot.pdf}
\end{minipage}



\end{longtable}

\begin{longtable}{p{3.2cm}| p{11cm}}
\texttt{url\_info} &\textbf{URL with additional info on the party}\newline 
URL to informaton on the party on the web. Can contain multiple URLs!
\end{longtable}

\begin{longtable}{p{3.2cm}| p{11cm}}
\texttt{party\_remarks\_
stelzle} &\textbf{Party remarks Stelzle}\newline 
Remarks on the party by me.
\end{longtable}

\begin{longtable}{p{3.2cm}| p{11cm}}
\texttt{party\_remarks\_
bundeswahlleiter} &\textbf{Party remarks Bundeswahlleiter}\newline 
Remarks on the party as listed by the Bundeswahlleiter.
\end{longtable}

\begin{longtable}{p{3.2cm}| p{11cm}}
\texttt{gueltige\_stimm
-zettel\_hh\_hb} &\textbf{Gültige Stimmzettel HH and HB}\newline 
Messy totals.
\end{longtable}

\begin{longtable}{p{3.2cm}| p{11cm}}
\texttt{gesamtstimmen\_by} &\textbf{Gesamtstimmen BY}\newline 
Messy totals.
\end{longtable}

\begin{longtable}{p{3.2cm}| p{11cm}}
\texttt{ausgefallene\_
stimmen\_be} &\textbf{Ausgefallene Stimmen BE}\newline 
Messy totals.
\end{longtable}

\begin{longtable}{p{3.2cm}| p{11cm}}
\texttt{abgegebene\_
stimmen\_hh} &\textbf{Abgegebene Stimmen HH}\newline 
Messy totals.
\end{longtable}

\begin{longtable}{p{3.2cm}| p{11cm}}
\texttt{ungueltige\_
stimmen\_except\_
hh\_hb} &\textbf{Ungültige Stimmen except in HH and HB}\newline 
Messy totals.
\end{longtable}

\begin{longtable}{p{3.2cm}| p{11cm}}
\texttt{ungueltige\_
stimmzettel\_hh\_hb} &\textbf{Ungültige Stimmzettel in HH and HB}\newline 
Messy totals.
\end{longtable}

\newpage

\hypertarget{ltw_election_results_and_gov}{%
\section{\texorpdfstring{\texttt{ltw\_election\_results\_and\_gov}}{ltw\_election\_results\_and\_gov}}\label{ltw_election_results_and_gov}}

This codebook only concerns variables specific to the
\texttt{ltw\_election\_results\_and\_gov} dataset. For further variables
plese refer to the \texttt{ltw\_election\_results} dataset's codebook.

\texttt{ltw\_election\_results\_and\_gov} returns a returns data frame
(tibble if the tibble package is loaded) containing both election
results as well as linked information on governments in the German
states. Each row contains information on one party during the time in
office of one cabinet. For a schematic version of
\texttt{bundeslaendeR::ltw\_election\_results\_and\_gov}'s structure see
table\(~\)\ref{strltwelecgov}.

Election results data are provided by the Bundeswahlleiter. A
machine-readable version of the data in the pdf available here
(\url{https://www.bundeswahlleiter.de/service/landtagswahlen.html}) was
kindly provided to me. Election data outside the timeframe covered by
Bundeswahlleiter's data provided to me was collected from the states'
local election authorities' (Landeswahlleiter) websites. More
information on parties and the continuity of parties under different
labels was collected by me. Information on Governments mainly taken from
the replication data of Linhart, Eric, Franz U. Pappi und Ralf Schmitt
(2008): Die proportionale Ministerienaufteilung in deutschen
Koalitionsregierungen: Akzeptierte Norm oder das Ausnutzen strategischer
Vorteile?, Politische Vierteljahresschrift 49(1): 46-67. To be found
online here:
\url{https://www.tu-chemnitz.de/phil/politik/pspi/forschung/daten.php}.
Information outside the timeframe of Linhart et al.~as well as
information on the names and party affiliations of the
Ministerpräsidenten was collected by me, mainly from German Wikipedia.

\begin{landscape}

    \setcounter{table}{2}
\begin{table}

\normalsize
\caption{Structure of \texttt{ltw\_election\_results\_and\_gov}}
\label{strltwelecgov}
\centering
\tiny
\texttt{
    \begin{tabular}{llrllrllrllrllrllr}
        \toprule
        \multicolumn{3}{c}{\textbf{\textsf{State Variables}}} & \multicolumn{3}{c}{\textbf{\textsf{Election Variables}}} & \multicolumn{3}{c}{\textbf{\textsf{Party Variables}}} & \multicolumn{3}{c}{\textbf{\textsf{Party-Election Variables}}} &
        \multicolumn{3}{c}{\textbf{\textsf{Government Variables}}} &
        \multicolumn{3}{c}{\textbf{\textsf{Government-Party Variables}}} \\
        \multicolumn{3}{c}{\textsf{Name, Abbreviation,}} & \multicolumn{3}{c}{\textsf{Election date, Size}} & \multicolumn{3}{c}{\textsf{Names, Abbreviations, several IDs}} & \multicolumn{3}{c}{\textsf{Vote Count, -Share, Seat}}&
        \multicolumn{3}{c}{\textsf{Inauguration date, PM Name, gov.}} &
        \multicolumn{3}{c}{\textsf{Status in government, number of}}\\
        \multicolumn{3}{c}{\textsf{NUTS1 Code}} &
        \multicolumn{3}{c}{\textsf{Electorate, Turnout, ...}} &
        \multicolumn{3}{c}{\textsf{several IDs}} &
        \multicolumn{3}{c}{\textsf{Count, -Share, ...}} &
        \multicolumn{3}{c}{\textsf{numbering, gov\_id, ...}} &
        \multicolumn{3}{c}{\textsf{party ministers, ...}} \\
        \cmidrule(r){1-3}\cmidrule(lr){4-6}\cmidrule(lr){7-9}
        \cmidrule(lr){10-12}\cmidrule(lr){13-15}\cmidrule(l){16-18}
        \textbf{state} & \textbf{nuts1} & \textbf{...} & \textbf{election\_date} & \textbf{turnout} & \textbf{...} & \textbf{partyname\_short} & \textbf{ches\_id} & \textbf{...} & \textbf{party\_vshare} & \textbf{party\_seat\_count} & \textbf{...} & \textbf{gov\_start\_date} & \textbf{minister\_president} & \textbf{...} & \textbf{gov\_party} & \textbf{nmin\_party} & \textbf{...} \\
        \midrule
        BE & DE3 & ... & 2015-09-18 & 0.765 & ... & Party A & 001 & ... & 0.45 & 46 & ... & 2015-10-07 & Mustermann, Max & ... & TRUE & 7 & ... \\
        BE & DE3 & ... & 2015-09-18 & 0.765 & ... & Party B & 002 & ... & 0.30 & 12 & ... & 2015-10-07 & Mustermann, Max & ... & TRUE & 4 & ... \\
        BE & DE3 & ... & 2015-09-18 & 0.765 & ... & Party C & 003 & ... & 0.25 & 18 & ...& 2015-10-07 & Mustermann, Max & ... & FALSE & NA & ... \\
        &&&&&&&&&&&\\[-2ex]
        BE & DE3 & ... & 2015-09-18 & 0.765 & ... & Party A & 001 & ... & 0.45 & 46 & ... & 2017-02-28 & Mustermann, Max & ... & TRUE & 11 & ... \\
        BE & DE3 & ... & 2015-09-18 & 0.765 & ... & Party B & 002 & ... & 0.30 & 12 & ... & 2017-02-28 & Mustermann, Max & ... & FALSE & NA & ...\\
        BE & DE3 & ... & 2015-09-18 & 0.765 & ... & Party C & 003 & ... & 0.25 & 18 & ... & 2017-02-28 & Mustermann, Max & ... & FALSE & NA & ...\\
        &&&&&&&&&&&\\
        NI & DE9 & ... & 2012-12-16 & 0.560 & ... & Party A & 001 & ... & 0.17 & 12 & ... & 2013-01-07 & Musterfrau, Erika & ... & FALSE & NA & ...\\
        NI & DE9 & ... & 2012-12-16 & 0.560 & ... & Party B & 002 & ... & 0.33 & 27 & ... & 2013-01-07 & Musterfrau, Erika & ... & FALSE & NA & ...\\
        NI & DE9 & ... & 2012-12-16 & 0.560 & ... & Party D & 004 & ... & 0.50 & 46 & ... & 2013-01-07 & Musterfrau, Erika & ... & TRUE & 13 & ...\\
        \bottomrule
\end{tabular}}
\end{table}

\end{landscape}

\hypertarget{governments-variables}{%
\subsection{Governments Variables}\label{governments-variables}}

\begin{longtable}{p{3.2cm}| p{11cm}}
\texttt{gov\_no\_within\_
legterm} &\textbf{Number of cabinet within legislative term}\newline 
Number of cabinet within legislative term (i.e. First cabinet in the 1990-1994 legislative term of state X).



\hspace*{.25cm}
\begin{minipage}[t]{\linewidth }
\vspace{0pt}
\includegraphics[width = \linewidth]{cbfiles/govnolegperplot.pdf}
\end{minipage}



\end{longtable}

\begin{longtable}{p{3.2cm}| p{11cm}}
\texttt{gov\_id} &\textbf{Government ID}\newline 
Unique ID of government. Taken from Linhart et al. However, this ID is not counting up within state by time. In cases where Governments were 
           missing from Linhart et al. before the timeframe covered by Linhart et al. (eg. in Berlin) these earlyer governments have an higher ID than later cabinets 
           contained in Linhart et al. data.
\end{longtable}

\begin{longtable}{p{3.2cm}| p{11cm}}
\texttt{state\_gov\_
number} &\textbf{Number of government in state.}\newline 
Number of government in state.



\hspace*{.25cm}
\begin{minipage}[t]{\linewidth }
\vspace{0pt}
\includegraphics[width = \linewidth]{cbfiles/stategovnumplot.pdf}
\end{minipage}



\end{longtable}

\begin{longtable}{p{3.2cm}| p{11cm}}
\texttt{gov\_party} &\textbf{Government Party}\newline 
Boolean wether the party was a cabinet party. Note: There is a single cabinet where no party is marked as part of the cabinet: Heinrich Welsch's caretaker government in the Saarland (at the time not yet a member of the FRG) in 1955.
\end{longtable}

\begin{longtable}{p{3.2cm}| p{11cm}}
\texttt{nmin\_party} &\textbf{Number of Ministers of Party}\newline 
Number of Ministers of Party.



\hspace*{.25cm}
\begin{minipage}[t]{\linewidth }
\vspace{0pt}
\includegraphics[width = \linewidth]{cbfiles/nminpartyplot.pdf}
\end{minipage}



\end{longtable}

\begin{longtable}{p{3.2cm}| p{11cm}}
\texttt{gov\_source} &\textbf{Government Source}\newline 
Source of the information on the government. Either Linhart et al. or the URL of the German Wikipedia Page containing information on the cabinet.
\end{longtable}

\begin{longtable}{p{3.2cm}| p{11cm}}
\texttt{gov\_remarks\_
stelzle} &\textbf{Governments remarks Stelzle}\newline 
My remarks on governments.
\end{longtable}

\begin{longtable}{p{3.2cm}| p{11cm}}
\texttt{minister\_president} &\textbf{Name of minister president}\newline 
Name of minister president.
\end{longtable}

\begin{longtable}{p{3.2cm}| p{11cm}}
\texttt{mp\_party} &\textbf{Minister President's Party}\newline 
Party of the minister president. partyname\_short format used. Note: There is a single cabinet with an independent minister president: Heinrich Welsch's caretaker government in the Saarland (at the time not yet a member of the FRG) in 1955.
\end{longtable}

\begin{longtable}{p{3.2cm}| p{11cm}}
\texttt{is\_mp\_party} &\textbf{Is MP Party?}\newline 
 the governments minister president from this party? Note: There is a single cabinet where the minister president is not part of any party: Heinrich Welsch's caretaker government in the Saarland (at the time not yet a member of the FRG) in 1955.
\end{longtable}

\newpage

\hypertarget{link_polidoc_parties-and-link_polidoc_governments}{%
\section{\texorpdfstring{\texttt{link\_polidoc\_parties} and
\texttt{link\_polidoc\_governments}}{link\_polidoc\_parties and link\_polidoc\_governments}}\label{link_polidoc_parties-and-link_polidoc_governments}}

\texttt{link\_polidoc\_parties} and \texttt{link\_polidoc\_governments}
provide easy links of \texttt{ltw\_election\_results} or
\texttt{ltw\_election\_results\_and\_gov} with party manifestos and
coalition agreements made available from polidoc.net - The Political
Documents Archive
\parencites{benoitChallengesEstimatingPolicy2009}{grossDoesEURegional2018}{pappiPolitikpositionenDeutschenLandtagsparteien2014}{pappiPartyElectionProgrammes2009}[for the codebook see][]{brauningerPolidocNetCodebook2018}.
While file names from polidoc.net follow a naming pattern
(\texttt{{partyID}.{stateID}.{year}.{1}.{number of party manifesto for election}}),
the provided links make joining the data easier.

Note that polidoc.net provides a manifesto for the Neue Liberale in the
HB 2015 election (41441.005.2015.1.1). Since the party withdrew it's
candidacy before the election and is thus not included in the election
results in \texttt{ltw\_election\_results}, the manifesto id is not
included in link\_polidoc\_parties.

Note that polidoc.net provides a coalition agreement between the SPD and
the Greens following the 2008 HE election (41001.006.2008.1.1). Since
this potential coalition under leadership of SPD politician Andrea
Ypsilanti never came to be due to several SPD MPs opposing the red-green
minority cabinet being externally supported by Die Linke the coalition
agreement can't be matched with a government in
\texttt{ltw\_election\_results\_and\_gov} and is thus not included.

\hypertarget{polidoc-link-variables}{%
\section{Polidoc Link Variables}\label{polidoc-link-variables}}

\begin{longtable}{p{3.2cm}| p{11cm}}
\texttt{state} &\textbf{State Abbreviation}\newline 
ISO 3166-2:DE-code of the state;
           including BA for the former state of Baden, WH for the former state
           of Württemberg-Hohenzollern and WB for the former state of
           Württemberg-Baden.
\end{longtable}

\begin{longtable}{p{3.2cm}| p{11cm}}
\texttt{election\_date} &\textbf{Election Date}\newline 
Date of the election.  ISO 8601 or R-Date format.
\end{longtable}

\begin{longtable}{p{3.2cm}| p{11cm}}
\texttt{polidoc\_filename and polidoc\_filename\_2 in link\_polidoc\_
parties} &\textbf{Polidoc File Name of Party Manifesto}\newline 
File name of state party manifesto (or 2nd manifesto if available) available in The Political Documents Archive (polidoc.net).


\hspace*{.25cm}
\begin{minipage}[t]{\linewidth }
\vspace{0pt}
\includegraphics[width = \linewidth]{cbfiles/pltpolidocparties.pdf}
\end{minipage}


\end{longtable}

\begin{longtable}{p{3.2cm}| p{11cm}}
\texttt{polidoc\_filename in link\_polidoc\_
governments} &\textbf{Polidoc File Name of Coalition Agreement}\newline 
File name of coalition agreement available in The Political Documents Archive (polidoc.net).


\hspace*{.25cm}
\begin{minipage}[t]{\linewidth }
\vspace{0pt}
\includegraphics[width = \linewidth]{cbfiles/pltpolidocgov.pdf}
\end{minipage}





\end{longtable}

\newpage

\hypertarget{de_states_grid_4x4}{%
\section{\texorpdfstring{\texttt{de\_states\_grid\_4x4()}}{de\_states\_grid\_4x4()}}\label{de_states_grid_4x4}}

\texttt{de\_states\_grid\_4x4()} exports a data frame containing state
IDs, german and english state names and approximate state locations on a
4x4 grid. The exported data frame can be used to approximately plot
state-facets in their approximate locations using \texttt{ggplot2}
extension \texttt{geofacet} \autocite{hafenGeofacetGgplot2Faceting2020}.

Please find a comparison of state locations and the grid layout below.

\hypertarget{example-code}{%
\subsubsection{Example Code:}\label{example-code}}

\begin{Shaded}
\begin{Highlighting}[]
\FunctionTok{library}\NormalTok{(bundeslaendeR)}
\FunctionTok{library}\NormalTok{(tidyverse)}
\FunctionTok{library}\NormalTok{(geofacet)}

\NormalTok{turnout\_plot }\OtherTok{\textless{}{-}}
\NormalTok{ltw\_election\_results }\SpecialCharTok{\%\textgreater{}\%} 
  \FunctionTok{select}\NormalTok{(state, election\_date, turnout) }\SpecialCharTok{\%\textgreater{}\%} 
  \FunctionTok{distinct}\NormalTok{() }\SpecialCharTok{\%\textgreater{}\%} 
  \FunctionTok{filter}\NormalTok{(}\SpecialCharTok{!}\NormalTok{(state }\SpecialCharTok{\%in\%} \FunctionTok{c}\NormalTok{(}\StringTok{"WB"}\NormalTok{, }\StringTok{"BA"}\NormalTok{, }\StringTok{"WH"}\NormalTok{))) }\SpecialCharTok{\%\textgreater{}\%} 
  \FunctionTok{filter}\NormalTok{(}\SpecialCharTok{!}\FunctionTok{is.na}\NormalTok{(turnout)) }\SpecialCharTok{\%\textgreater{}\%} 
  \FunctionTok{ggplot}\NormalTok{(}\FunctionTok{aes}\NormalTok{(}\AttributeTok{x =}\NormalTok{ election\_date, }\AttributeTok{y =}\NormalTok{ turnout)) }\SpecialCharTok{+}
    \FunctionTok{geom\_line}\NormalTok{() }\SpecialCharTok{+}
    \FunctionTok{facet\_geo}\NormalTok{(}\AttributeTok{grid =} \FunctionTok{de\_states\_geofacet\_grid\_4x4}\NormalTok{(}\AttributeTok{linebreak =}\NormalTok{ T),}
              \AttributeTok{facets =} \SpecialCharTok{\textasciitilde{}}\NormalTok{state, }\AttributeTok{label =} \StringTok{"name"}\NormalTok{) }\SpecialCharTok{+}
    \FunctionTok{scale\_y\_continuous}\NormalTok{(}\AttributeTok{limits =} \FunctionTok{c}\NormalTok{(}\DecValTok{0}\NormalTok{,}\DecValTok{1}\NormalTok{),}
                       \AttributeTok{labels =}\NormalTok{ scales}\SpecialCharTok{::}\NormalTok{percent) }\SpecialCharTok{+}
    \FunctionTok{theme}\NormalTok{(}\AttributeTok{strip.text =} \FunctionTok{element\_text}\NormalTok{(}\AttributeTok{face =} \StringTok{"bold"}\NormalTok{)) }\SpecialCharTok{+}
    \FunctionTok{labs}\NormalTok{(}\AttributeTok{x =} \ConstantTok{NULL}\NormalTok{, }\AttributeTok{y =} \StringTok{"Turnout"}\NormalTok{)}
\end{Highlighting}
\end{Shaded}

\nocite{runfolaGeoBoundariesGlobalDatabase2020}

\newpage

\begin{landscape}

\begin{figure}

{\centering \includegraphics{cbfiles/map-vs-grid-plot-1} 

}

\caption{Comparison of state location and grid layout}\label{fig:map-vs-grid-plot}
\end{figure}

\end{landscape}

\printbibliography[title=References]

\end{document}
